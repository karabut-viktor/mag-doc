\section{Similar code detection}

Simplest possible way to find code duplication would be line-by-line comparison of source lines.
However code can differ by an indentation, comments, variable names. 

Other approach is parse language construct into language-specific Abstract 
Syntax Trees (AST) and then compare each node by node using recursive similarity
definition, as it did Baxter in his work \cite{Baxter&al1998}:

\begin{math}
Similarity =  \frac{ 2 \cdot S}{2 \cdot S + L + R}
\end{math}

where S is a number of shared nodes. L is a number of different nodes in left subtree,
and R - in right subtree.

In this work we will use similar approach, but instead of working with language-specific AST we will translate it to
language-agnostic intermediate code representation: Abstract Code Trees (ACT).
ACT is a digest of program syntax structure.
Main reason of introducing ACT is to a give formal definition of 'code similarity' which allows us to reason about
effectiveness of our search methods.

Intermediate representation will allow to make language-agnostic search engine, so it would be possible to migrate to
another language.
Laconic by nature ACT will help to keep the search engine simple and effective. 

As a bonus at Java translation step we get rid of various Java 'sugar' constructs,
such as introduced in Java 6 'foreach' cycle for collection traversing \cite{gosling05}.

\subsection{Description of ACT}

ACT (Abstract Code Tree) is a intermediate representation of code.
ACT structure was inspired by Nielson's while language \cite{Nielson&Nielson93}.
ACT tree structure are describes these Extended Backus-Naur Form (EBNF)\cite{ISO1996} rules. 

\begin{verbatim}
  stmt = block | if | while | expr | try-catch| return | throw
  block = begin, { stmt }, end
  if = "if", expr, "then", block, "else", block
  while = "while", expr, "do", block
  try-catch = "try", block, {"catch ", str_literal, block}
  expr = assign | var | call | str_literal | num_literal
  assign = expr ":=" expr
  call = "f(", str_literal, {, expr}, ")"
  return = "return", expr
  throw = "throw", expr
\end{verbatim}

This is briefly description of all ACT nodes.
\begin{description}
  \item[Statement 'stmt'] \hfill \\
  Root class for all ACT nodes. Every node is statements.
  \item[Code block: 'block'] \hfill \\
  Block of code. 
  In while-language syntax Nielson uses concatenation(';') operator to chain multiply statements\cite{Nielson&Nielson93}.
  However introducing of concatenation operator doesn't fit well into our clone detection framework.
  Our engine should handle clones which have few statements inserted or removed from the middle of code.
  Using concatentation operators results to deep trees, where inserting to the end of block result to change of whole tree.
  Instead of this we use blocks which contains ordered lists of statements.
  This flattens tree structure and allows to apply different edit distance algorithms.
  \item[If statement 'if'] \hfill \\
  Conditional statement node.
  Digest of the Java if-statement.
  Additionally Java multiple choise conditional statements 'switch-case' are also translated to this node.
  Our conditional statement always contains 'else' branch even if it is an empty code block.
  \item[While cycle 'while'] \hfill \\
  Simple cycle with precondition.
  Digest of Java while-cycle.
  All kinds of Java iteration statements are translated to this node.
  Always contains body even if it is an empty code block.
  \item[Try-catch block 'try-catch'] \hfill \\
  Digest of Java try-catch block.
  \item[Expression 'expr'] \hfill \\
  Root class for all ACT expressions.
  Expression is a statement which have value.
  \item[Assignment statement 'assign'] \hfill \\
  Digest of Java '=' expression.
  According ACT syntax can contain any expression on both sides.
  We do not divide expression by right and left value in our engine.
  \item[Method call] \hfill \\
  Method call.
  Has wide variety of uses in our
  Digest of java dot operator.
  All Java static method calls, virtual method call and even constructor calls are translated to this node.
  Moreover this node is a digest of all Java binary operators, such as comparison '==', multiplication '*', addition '+' etc.
  \item[Return] \hfill \\
  Digest of Java return statement
  \item[Throw] \hfill \\
  Digest of Java throw statement.
  Used for throwing exceptions.
\end{description}

\subsection{Translation ACT}

ACT have much less constructs than Java. 
In some cases we express different Java language statements with same ACT constructs or, in rare cases, we just drop
them.
Our goal is to get ACT representation for our similar-code search engine, so we intentionally loose some Java
statements, like labels and 'goto' operators.
It means that two code fragments which differ only by one label will be considered identical.
This precision is enough for our practical purposes, because labels aren't widely used in modern Java software.

ACT doesn't have notion of unary or binary operators. 
So all basic arithmetic, logic and comparison operators are translated to special method call, for example Java
expression '3*(2+1)' will be translated to 'f(*,3,f(+,2,1)'.

ACT if-statement is same as in Java except it always have else branch.
Java switch-case statement are translated to series of consecutive if-statements.
Translator expects that every case block are followed by 'break' statement.
If break statement is absent translator would add it artificially, it isn't semantically correct behaviour but
good-enough for our similar code detection. \\

\renewcommand{\arraystretch}{0.01}
\begin{tabular}{ p{6cm} p{6cm} }
\textbf{Java switch statement} & \textbf{ACT} \\
\begin{verbatim}
switch(a): {
  case 1:
  	doA(); break;
  case 2:
    doB(); break;
  default:
  	doC(); break;
}
\end{verbatim} 
  &
\begin{verbatim}
if (f('eq', a, 1) then 
  begin f('doA') end
else 
  begin
    if (f('eq', b, 1) then
      begin f('doB') end 
    else
      begin f('doC') end
  end
\end{verbatim}
\end{tabular}


Java language has 4 different iteration statements\cite{gosling05}. Whyle-cycle translates as is. Java do-cycle converted to
while by repeating the cycle body. For-cycle initialization moves outside of cycle and counter expression are
appended to the end of body.\\

\begin{tabular}{ p{6cm} p{6cm} }
\textbf{Java while cycle} & \textbf{ACT} \\
\begin{verbatim}
while(code()) {
  doA();
}
\end{verbatim} 
  & 
\begin{verbatim}
while f('cond') do begin
  f('doA')
end
\end{verbatim}
\\~\\
\end{tabular}

\begin{tabular}{ p{6cm} p{6cm} }
\textbf{Java do-while cycle} & \textbf{ACT} \\
\begin{verbatim}
do {
  doA();
} while (cond());
\end{verbatim} 
  & 
\begin{verbatim}
doA();
while f('cond') do begin
  f('doA')
end
\end{verbatim}
\\~\\
\end{tabular}

\begin{tabular}{ p{6cm} p{6cm} }
\textbf{Java for cycle} & \textbf{ACT} \\
\begin{verbatim}
for (init(); cond(); counter()){
  doA();
}
\end{verbatim} 
  & 
\begin{verbatim}
f('init')
while f('cond') do begin
  f('doA')
  f('counter')
end
\end{verbatim}
\end{tabular}

Added in Java 6 enhanced for-cycle is a syntax sugar for Java SE collection framework API \cite{oracle2015oracle}.
We replace them with equivalent for-cycle and then translate it to ACT while.
For example these two Java code fragment are equivalent in sense of generated bytecode:\\

\begin{tabular}{ p{6cm} p{6cm} }
\textbf{Java source code} & ~ \\
\begin{verbatim}
for (X x : xs){
  do(x);
}
\end{verbatim} 
  & 
~
\end{tabular}

\begin{tabular}{ p{6cm} p{6cm} }
\textbf{Equivalent Java code} & \textbf{ACT} \\
\begin{verbatim}
for (
    Iterator it = xs.iterator();
    it.hasNext();
    ) {
  X x = (X) it.next();
  do(x);    
}
\end{verbatim} 
  & 
\begin{verbatim}
it = f('java.util.Collection#iterator', xs)
while f('java.util.Iterator#hasNext', it) do 
begin
  x = f('java.util.Iterator#next', it)
  f('do', x)
end
\end{verbatim}
\end{tabular}

Some statements are completely dropped during translation: different 'goto' statements, assertions and synchronization
blocks. 'Finally' blocks isn't  present on ACT, but its content added just after try-catch statement.

\subsection{Definition of similarity}

How close are two code fragments?
Are they clones or not?
To answer this question we define similarity between two ACT nodes is a float number in an interval from 0 to 1.
Similarity should be equal 1 if code is completely identical, and 0 if have nothing in common.
Now we can define similarity between every possible ACT nodes.

\begin{align*}
  Sim \colon & ACT \times ACT \mapsto \lbrack 0 \ldots 1 \rbrack
\end{align*}

First and important rule: similarity of two nodes with different types are always 0.
\begin{align*}
  Sim(x, y) = 0 \text{ if $type(x) \neq type(y)$ }
\end{align*}

Now we should define our similarity function for every node type. Let's start fro numerical literals. Numerical literals are similar if and only if they are values are equal.

\begin{align*}
Sim(num(x), num(y)) &= 
	\begin{cases} 
		\hfill 1  \hfill & \text{ if $x = y$} \\
		\hfill 0	\hfill & \text{ otherwise } \\
	\end{cases}
\end{align*}

For string literals good candidate is the Levenshtein edit distance\cite{surveys/navarro01}.
The Levenshtein distance between two words shows the minimum number of single-character insertions, deletions or substitutions required to change one string into the another.
To get value in needed interval, we normalize output of Levenshtein function by dividing to string length.

\begin{align*}
Sim(s_a, s_b) &= 1 - \frac{lev(s_a,s_b)}{max(length(s_a), length(s_b))}
\end{align*}

Using same idea we can define editional distance for two code blocks: edit distance of two code blocks is the minimum
number of single statement insertions, deletions or substitutions required to change one code block into other.
Insertions and deletions are count as 1 operation, substitutions cound as $1 - sim(stm_a, stm_b)$ operations.
Such approach will help us to handle similar code fragment where only few lines of code were removed or added.

\begin{align*}
	Sim(block_a, block_b) &= 1 - \frac{lev(block_a,block_b)}{max(length(block_a), length(block_b))}
\end{align*}

Two method call similarities a computed by same rules as a block similarity, but instead of block statements we compute
distance of argument expressions.

\begin{align*}
	Sim(f_a, f_b) &= 1 - \frac{lev(f_a,f_b)}{max(length(f_a), length(f_b))}
\end{align*}

We define similarity of 'if' node as a linear composition of similarities of child nodes. 

\begin{align*}
	Sim(&if(expr_a, block_{a,then}, block_{a,else}), if(expr_b, block_{b,then}, block_{b,else})))\\
	=& C_{if,0} \\
	+& C_{if,1} Sim(expr_a, expr_b) \\
	+& C_{if,2} Sim(block_{a,then}, block_{b,then}) \\
	+& C_{if,3} Sim(block_{a,else}, block_{b,else})
\end{align*}

Constant $C_{if,0}, C_{if,1}, C_{if,2}, C_{if,3}$ can be selected empirically such as $C_{if,0}, C_{if,1}, C_{if,2},
C_{if,3} >= 0$ and $C_{if,0} + C_{if,1} + C_{if,2} + C_{if,3} = 1$ to ensure that similarity always in interval 0..1.
Current version of ACT engine uses following constants $C_{if,1} = 0.1$, $C_{if,1} = 0.2$, $C_{if,1} = 0.4$, $C_{if,3} = 0.3$.
For example if two 'if' statements have identical code blocks, but completely different conditional expressions such as $Sim(e_a,e_b) = 0$ their
similarity can be computed as following:

\begin{align*}
	Sim(&if(e, b_{then}, b_{else}), if(e, b_{then}, b_{else}))) = \\
	=& C_{if,0} + C_{if,1} Sim(e_a, e_b) + C_{if,2} Sim(b_{then}, b_{then}) + C_{if,3} Sim(b_{else}, b_{else}) \\
	=& 0.1 + 0.2 Sim(e_a, e_b) +	0.4 Sim(b_{then}, b_{then}) +	0.3 Sim(b_{else}, b_{else}) \\
	=& 0.1 + 0 + 0.4 + 0.3 \\
	=& 0.8
\end{align*}

Same approach we use to define similarity of 'while' nodes:

\begin{align*}
	Sim(&while(expr_a, block_a), while(expr_b, block_b)) = \\
	=&C_{while,0} \\
	+&C_{while,1} Sim(expr_a, expr_b) \\
	+&C_{while,2} Sim(block_a, block_b)
\end{align*}




