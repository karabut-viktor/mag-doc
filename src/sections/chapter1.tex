\section{Similar code detection}

Simplest possible way to find code duplication would be line-by-line comparison of
source lines. However code can differ by an indentation, comments, variable
names. 

Other approach is parse language construct into language-specific Abstract 
Syntax Trees (AST) and then compare each node by node using recursive similarity
definition, as it did Baxter in his work \cite{Baxter&al1998}:

\begin{math}
Similarity =  \frac{ 2 \cdot S}{2 \cdot S + L + R}
\end{math}

where S is a number of shared nodes. L is a number of different nodes in left subtree,
and R - in right subtree.

In this work we will use similar approach, but instead of working with
language-specific AST we will translate it to language-agnostic intermediate
code representation: Abstract Code Trees (ACT).  ACT is laconic imperatice 
language. ACT doesn't meant to be executed, althroug this possible. Main reason
of introducing ACT is a give formal definition of 'code similarity' which allows
us to reason about effectiveness of our search methods. 

Intermediate representation will allow to make language-agnostic search engine, so it would be possible to migrate to
another language. Laconic by nature ACT will help to make search engine simple
and effective. 

As a bonus at Java translation step we get rid of various Java 'sugar' constructs,
such as introduced in Java 6 'foreach' cycle for collection traversing. All possible 

Not all Java constructs are translated to ACT. Some drop

\subsection{Description of ACT}

ACT (Abstract Code Tree) is a intermediate representation of code. ACT was inspired
by Nielson's while language (cite?). ACT doesn't meant to be executed, however
execution is also possible. Language describes these EBNF rules:

\begin{verbatim}
stmt = block | if | while | expr | try-catch| return | throw
block = begin, { stmt }, end
if = "if", expr, "then", block, "else", block
while = "while", expr, "do", block
expr = assign | var | call | str_literal | num_literal
assign = expr ":=" expr
call = "f(", str_literal, {, expr}, ")"
try-catch = "try", block, {"catch ", str_literal, block}
return = "return", expr
throw = "throw", expr
\end{verbatim}

\subsection{Translation ACT}

ACT have much less constructs than Java. In some cases we express different Java language statements with same ACT
constructs, or, in rare cases, we just drop them. Our goal is to get ACT representation for our
similar-code search engine, so we intentionally loose some Java semantics, like 'goto' statements.

ACT doesn't have notion of unary or binary operators, so all basic arithmetic, logic and comparison operators 
are translated to special method call, for example Java expression '3*(2+1)' will be translated to 'f(*,3,f(+,2,1)'.

ACT if-statement is same as in Java except it always have else branch. Java switch-case statement are translated to
series of consecutive if-statements. Translator expects that every case block are followed by 'break' statement.
If break statement is absent translator would add it artificially, it isn't semantically correct behaviour but
good-enough for our similar code detection.

\begin{tabular}{ p{6cm} p{6cm} }
\begin{verbatim}
switch(a): {
  case 1:
  	doA(); break;
  case 2:
    doB(); break;
  default:
  	doC(); break;
}
\end{verbatim} 
  & 
\begin{verbatim}
if (f('eq', a, 1) then 
  begin f('doA') end
else 
  begin
    if (f('eq', b, 1) then
      begin f('doB') end 
    else
      begin f('doC') end
  end
\end{verbatim}
\end{tabular}

Java language has 4 different cycle statements. Whyle-cycle translates as is. Java do-cycle converted to
while by repeating the cycle body. For-cycle initialization moves outside of cycle and counter expression are
appended to the end of body.

\begin{tabular}{ p{6cm} p{6cm} }
\begin{verbatim}
do {
  doA();
} while (cond());
\end{verbatim} 
  & 
\begin{verbatim}
doA();
while f('cond') do begin
  f('doA')
end
\end{verbatim}
\end{tabular}

\begin{tabular}{ p{6cm} p{6cm} }
\begin{verbatim}
for (init(); cond(); counter()){
  doA();
}
\end{verbatim} 
  & 
\begin{verbatim}
f('init')
while f('cond') do begin
  f('doA')
  f('counter')
end
\end{verbatim}
\end{tabular}

Added in Java 6 enhanced for-cycle is a syntax sugar for Java SE collection framework API. We replace them
with equivalent for-cycle and then translate it to ACT while. For example these two Java code fragment are equivalent in
sense of generated bytecode:

\begin{tabular}{ p{6cm} p{6cm} }
\begin{verbatim}
for (X x : xs){
  do(x);
}
\end{verbatim} 
  & 
\begin{verbatim}
for (Iterator it = xs.iterator(); it.hasNext();) {
  X x = (X) it.next();
  do(x);    
}
\end{verbatim}
\end{tabular}

Some statements are completely dropped during translation: different 'goto' statements, assertions and synchronization
blocks. 'Finally' blocks isn't  present on ACT, but its content added just after try-catch statement.

\subsection{Definition of similarity}

We define similarity between two ACT nodes is a float in interval from 0 to 1.
Similarity should be equal 1 if ACT is identical, and be 0 if we compare nodes
of different types.
Now we can define similarity recursively for every node.

Two numerical literals are similar if and only if they are equal:

\begin{align*}
Sim(x_a, x_b) &= 
	\begin{cases} 
		\hfill 1    \hfill & \text{ if $x_a = x_b$} \\
		\hfill 0	\hfill & \text{ otherwise} \\
	\end{cases}
\end{align*}

For string literals good candidate is Levenshtein edit distance. The 
Levenshtein distance between two words is the minimum number of single-character insertions, deletions or
substitutions required to change one string into the other (cite?). To get value in needed interval, we will
normalize output of Levenshtein function by dividing to string length.

\begin{align*}
Sim(s_a, s_b) &= 1 - \frac{lev(s_a,s_b)}{max(length(s_a), length(s_b))}
\end{align*}

Using same idea we can define editional distance for two code blocks:
edit distance of two code blocks if the minimum number of single statement insertions, deletions or
substitutions required to change one code block into other. Insertions and deletions are count as 1 operation,
substitutions cound as $1 - sim(stm_a, stm_b)$ operations. Such approach will help to handle similar code fragment where
only few lines were removed or added.

\begin{align*}
	Sim(block_a, block_b) &= 1 - \frac{lev(block_a,block_b)}{max(length(block_a), length(block_b))}
\end{align*}

We define similarity of 'if' node as a linear composition of similarities of child nodes. 

\begin{align*}
	Sim(if(expr_a, block_{a,then}, block_{a,else}), if(expr_b, block_{b,then}, block_{b,else}))) &= \\
	&C_{if,0} + \\
	&C_{if,1} Sim(expr_a, expr_b) + \\
	&C_{if,2} Sim(block_{a,then}, block_{b,then}) + \\
	&C_{if,3} Sim(block_{a,else}, block_{b,else})
\end{align*}

Constant $C_{if,0}, C_{if,1}, C_{if,2}, C_{if,3}$ can be selected empirically such as $C_{if,0}, C_{if,1},
C_{if,2}, C_{if,3} >= 0$ and $C_{if,0} + C_{if,1} + C_{if,2} + C_{if,3} = 1$ to ensure that similarity always in interval 0..1.

Same approach we use to define similarity of 'while' nodes:

\begin{align*}
	Sim(while(expr_a, block_a), while(expr_b, block_b)) &= \\
	&C_{while,0} + \\
	&C_{while,1} Sim(expr_a, expr_b) + \\
	&C_{while,2} Sim(block_a, block_b)
\end{align*}




