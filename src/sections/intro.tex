\section*{Introduction}
\addcontentsline{toc}{section}{Introduction}

Supreme goal of my work is an implementation of IDE plugin which check writed code and gives suggestion about code
organization. For example some similar routines from different modules of large project can be refactored 
into one utility method. Another usage scenario is a code fragments which can be replaced with utilities available from
some existing library. Often code reuse is limited by legal
issues, such as patents, copyrights. However there is a wide variety of an open source
projects, from where code can be used as an inspiration or even be directly
used.

Our goal is  to find not intentionally copied code, but code fragment which is similar because they
implements similar algorithms or similar code routines. In many previous works (cite?) code clones considered
as textual copies of code with changed formatting, added comments or renamed variables. 
In this article we emphasise on semantically similar code, which cannot even have similar statements, but which does
semantically same work. For example for every while-cycle we can write semantically equivalent while-cycle or replace
if-else with switch-case statement.

In the first chapter we consider questions about code similarity. In order to detect similar code
fragments we will parse Java language code into AST.
After that we introduce intermediate representation of parsed code: Abstract Code Trees (ACT).
During translation to ACT we get rid of reduntant for our purpose information.

In second chapter we will consider fast search engine for code fragments. ACT
each-to-each comparison complexity is $O(n^2 \cdot m^2)$ where $n$ is number of
trees and $m$ is size of each treee. For large code fragment database speed of
quadratic algorithms is unacceptable. To solve this issue we use a
hashing mechanism. For this we use fingerprinting approach.
Fingerprint is a generalized version of ACT, where removed some information. Fingerprinting allows to find in
constant time get some relatively small set of matching ACT from where we can find needed code fragments using an
quadratic algorithms.

Finally in third chapter we describe experiments with finding code fragments in
open source projects fetched from Maven Central database.
