\section*{Introduction}
\addcontentsline{toc}{section}{Introduction}
Currently there is a lot of code written in Java. Often developers waste time on
writing code which is already exists in another project or library. Our goal is
develop a tool, which would hint developers, where is already exist a similliar
code and how to improve it's quality. Often code reuse is limited by legal
issues, such as patents, copyrights. Hovewer there is a wide variety of an open source
projects, from where code can be used as an inspiration or even be directly
copied.

In first chapter we consider questions about code similarity. To detect similar
code fragmenst we will parse code into Abstract Syntax Trees (AST). After that
we can compute edit distance between AST's. Edit distance shows how many nodes
from original AST need to remove or add to get target AST. This is analogue of
Levenshteins distance for trees.

In second chapter we will consider search engine for code fragments. AST
each-to-each comparison complexity is $O(n^2 \cdot m^2)$ where $n$ is number of
trees and $m$ is size of each treee. For large code fragment database speed of
quadratic alghorithms is unacceptable. To solve this issue we use a
hashing mechanism. Unfortunately hashing of raw ASTs cannot help in finding similar code
fragment. For this we introduce Approximate Code Trees (ACT) fingerprints.
ACT is generalized version of AST, where some nodes replaced by patterns. Every ACT
represent wide class of ASTs. Now we can in constant time find matching ACT and
get some relatively small set of matching AST from where we can find more
similar code fragments using a slow quadratic alghorithms.

Finally in third chapter we describe experiments with finding code fragments in
open source projects fetched from Maven Central database.
